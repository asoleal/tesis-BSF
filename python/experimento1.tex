\documentclass[12pt, a4paper]{article}

% Paquetes para idioma y codificación
\usepackage[utf8]{inputenc}
\usepackage[spanish, es-tabla]{babel}

% Paquetes matemáticos y gráficos
\usepackage{amsmath}
\usepackage{amsfonts}
\usepackage{amssymb}
\usepackage{graphicx}
\usepackage{geometry}
\usepackage{float}

% Paquetes para código fuente
\usepackage{listings}
\usepackage{xcolor}

% Configuración de márgenes
\geometry{left=2.5cm, right=2.5cm, top=2.5cm, bottom=2.5cm}

% Configuración de estilo para código Python
\definecolor{codegreen}{rgb}{0,0.6,0}
\definecolor{codegray}{rgb}{0.5,0.5,0.5}
\definecolor{codepurple}{rgb}{0.58,0,0.82}
\definecolor{backcolour}{rgb}{0.96,0.96,0.96}

\lstdefinestyle{mystyle}{
    backgroundcolor=\color{backcolour},
    commentstyle=\color{codegreen},
    keywordstyle=\color{magenta},
    numberstyle=\tiny\color{codegray},
    stringstyle=\color{codepurple},
    basicstyle=\ttfamily\footnotesize,
    breakatwhitespace=false,
    breaklines=true,
    captionpos=b,
    keepspaces=true,
    numbers=left,
    numbersep=5pt,
    showspaces=false,
    showstringspaces=false,
    showtabs=false,
    tabsize=2,
    language=Python
}

\lstset{style=mystyle}

% Título del documento
\title{\textbf{Caracterización de la Dinámica Metabólica en BSF: Transición Aerobia-Anaerobia y Definición de Límites para PINNs}}
\author{Análisis Experimental - Tesis BSF}
\date{\today}

\begin{document}

\maketitle

\section{Introducción}
La precisión de las Redes Neuronales Informadas por la Física (PINNs) depende de la validez de las ecuaciones diferenciales subyacentes. El modelo de crecimiento de \textit{Hermetia illucens} asume una estequiometría aerobia ($O_2 \rightarrow CO_2$). Sin embargo, en sistemas de alta densidad larvaria, el agotamiento de oxígeno puede inducir rutas metabólicas anaerobias que violan las asunciones del modelo.

Este capítulo analiza un experimento de "choque alimenticio" tras 48 horas de ayuno para determinar:
\begin{enumerate}
    \item La tasa metabólica máxima ($V_{max}$) post-ingesta (Efecto SDA).
    \item El punto crítico de transición a la anaerobiosis (producción de metano).
    \item El intervalo de tiempo válido para el entrenamiento de la red neuronal.
\end{enumerate}

\section{Metodología de Procesamiento de Datos}
Se registraron las concentraciones de $CO_2$ y $CH_4$ en una cámara estática con 700 larvas y 250g de sustrato fresco. Para alinear temporalmente las series de tiempo y detectar automáticamente el inicio de la fermentación, se implementó el siguiente algoritmo en Python.

\begin{lstlisting}[caption={Algoritmo para la detección de umbrales metabólicos y limpieza de datos}, label={lst:python_script}]
import pandas as pd
import matplotlib.pyplot as plt
import numpy as np

# 1. Carga de Datos Crudos
# Archivos generados el 9 de Septiembre (Post-Ayuno)
df_co2 = pd.read_csv('experimento_co2.csv', sep='\t', header=None, names=['Hora', 'CO2_ppm'])
df_ch4 = pd.read_csv('experimento_ch4.csv', sep='\t', header=None, names=['Hora', 'CH4_ppm'])

# Creacion de Timestamp para sincronizacion
base_date = '2025-09-09'
df_co2['Timestamp'] = pd.to_datetime(base_date + ' ' + df_co2['Hora'])
df_ch4['Timestamp'] = pd.to_datetime(base_date + ' ' + df_ch4['Hora'])

# 2. Fusion (Merge) y Normalizacion Temporal
df = pd.merge(df_co2, df_ch4, on='Timestamp', suffixes=('_co2', '_ch4'))
t0 = df['Timestamp'].iloc[0]
df['Minutos'] = (df['Timestamp'] - t0).dt.total_seconds() / 60.0

# 3. Deteccion de Fase Anaerobia
# Se identifica el primer instante donde CH4 > 0
corte_anaerobio = df[df['CH4_ppm'] > 0]

if not corte_anaerobio.empty:
    t_corte = corte_anaerobio.iloc[0]['Minutos']
    co2_limite = corte_anaerobio.iloc[0]['CO2_ppm']
else:
    t_corte = df['Minutos'].max()

# 4. Calculo de Pendiente (Solo Fase Aerobia Valida)
# Se descartan los primeros 2 min (mezcla) y todo lo posterior al metano
fase_aerobia = df[(df['Minutos'] > 2) & (df['Minutos'] < t_corte)]
slope, intercept = np.polyfit(fase_aerobia['Minutos'], fase_aerobia['CO2_ppm'], 1)

print(f"Tasa de Produccion de CO2: {slope:.2f} ppm/min")
print(f"Inicio de Fermentacion (Corte): {t_corte:.2f} min")
\end{lstlisting}

\section{Resultados y Discusión}

El análisis de la dinámica de gases revela tres fases metabólicas distintas, tal como se muestra en la Figura \ref{fig:transicion}.

% IMPORTANTE: Asegúrate de tener la imagen 'grafica_transicion.png' en la misma carpeta
\begin{figure}[H]
    \centering
    \includegraphics[width=0.9\textwidth]{grafica_transicion.png}
    \caption{Dinámica de gases en tiempo real. La línea azul representa la respiración celular ($CO_2$), mientras que la naranja indica fermentación metanogénica ($CH_4$). La línea roja vertical marca el límite de validez para el modelo PINN.}
    \label{fig:transicion}
\end{figure}

\subsection{Fase 1: Latencia y Metabolismo Basal ($0 < t < 2$ min)}
Inicialmente, se observa un incremento marginal de $CO_2$ ($\approx 27$ ppm en 2 min). Este periodo corresponde a la aclimatación de las larvas y la mezcla de gases en el espacio de cabeza. Biológicamente, representa el término de mantenimiento ($m$) en ausencia de asimilación activa.

\subsection{Fase 2: Ventana Aerobia ($2 < t < 8$ min)}
Se registra un incremento lineal abrupto de $CO_2$ con una pendiente aproximada de $900 \text{ ppm/min}$. Este fenómeno se atribuye a la \textbf{Acción Dinámica Específica (SDA)}, el costo energético asociado a la ingesta y procesamiento inmediato del alimento tras el ayuno. Durante esta ventana, la concentración de $CH_4$ permanece en 0 ppm, confirmando condiciones aerobias.

\textbf{Implicación para la Tesis:} Estos son los únicos datos válidos para entrenar la PINN. La linealidad confirma que la tasa metabólica es constante mientras exista oxígeno disponible.

\subsection{Fase 3: Colapso Aerobio ($t > 8$ min)}
Al superar las $5600$ ppm de $CO_2$, el sistema entra en hipoxia crítica. Inmediatamente, el sensor detecta $CH_4$, indicando que la microbiota del sustrato y el tracto digestivo de las larvas han cambiado a metabolismo fermentativo.

\section{Conclusión para el Modelo Matemático}
Para garantizar la convergencia de la red neuronal y evitar el aprendizaje de dinámicas espurias, se establece el siguiente criterio de filtrado (\textit{Data Pruning}):
$$ \mathcal{D}_{train} = \{ (t, CO_2) \mid CH_4(t) = 0 \land t > 2 \text{ min} \} $$
Cualquier dato fuera de este rango introduciría error en la estimación de biomasa, ya que la producción de $CH_4$ representa una pérdida de carbono no contabilizada en el modelo de crecimiento estándar.

\end{document}
