\chapter{Antecedentes}

A continuación, se presentan los principales antecedentes relacionados con el estudio de esta propuesta doctoral, enfocados en la problemática de gestión de residuos, el potencial de soluciones biotecnológicas y, crucialmente, una revisión crítica sobre el papel y las limitaciones de las tecnologías emergentes en la mitigación del impacto ambiental.

\section{Problemática global en el manejo de residuos y la generación de emisiones de GEI}

La gestión inadecuada de residuos constituye una de las principales amenazas ambientales a escala global, con efectos directos sobre el cambio climático, la salud pública y la integridad de los ecosistemas. Según la FAO, se estima que más del 50\,\% de los residuos sólidos generados en países de ingresos medios y bajos son de origen orgánico \parencite{FAO2021}, los cuales, al ser dispuestos sin tratamiento adecuado, se degradan en condiciones anaeróbicas generando metano (\ce{CH4}). Este fenómeno es particularmente grave en sectores como la agricultura, donde las prácticas tradicionales —como la quema de residuos y el uso intensivo de insumos químicos— agravan la carga ambiental de los sistemas productivos.

De acuerdo con el Quinto Informe del Panel Intergubernamental sobre Cambio Climático, aproximadamente el 24\,\% de las emisiones globales de GEI provienen de actividades relacionadas con el uso del suelo, la agricultura y los cambios en el uso de la tierra \parencite{IPCC2014}. A estas emisiones se suman los impactos generados por el manejo ineficiente de residuos en contextos urbanos e industriales, incluyendo lixiviación contaminante, contaminación atmosférica y afectación de la biodiversidad local. Estas condiciones se ven agravadas por la limitada infraestructura de tratamiento y las brechas en gobernanza ambiental, especialmente en regiones rurales y en países en desarrollo.

Este contexto exige de manera urgente un diseño de estrategias sostenibles de gestión de residuos, que integren enfoques sistémicos y soluciones basadas en principios de economía circular. El enfoque de “residuo como recurso” promueve la valorización de la materia orgánica a través de tecnologías como el compostaje, la digestión anaerobia, la vermicultura y, más recientemente, la bioconversión con insectos. Estos enfoques permiten cerrar ciclos de nutrientes, reducir emisiones y generar productos útiles como biofertilizantes, energía o proteínas alternativas, entre otros.

\section{Tecnologías emergentes: Potencialidades y análisis crítico de riesgos}

Para superar las barreras de medición y control en estos bioprocesos, la literatura reciente sugiere la integración de sistemas de gestión inteligentes, capaces de monitorear en tiempo real la generación de emisiones. Esto implica la incorporación de tecnologías como sensores de Internet de las Cosas (IoT), modelos matemáticos y algoritmos de aprendizaje automático, herramientas que prometen mejorar la eficiencia y garantizar la trazabilidad basada en datos \parencite{ahmedAgricultureClimateChange2020}.

Sin embargo, la adopción de estas tecnologías no está exenta de riesgos y limitaciones que deben ser analizados críticamente para evitar soluciones contraproducentes (“tecnosolucionismo”). Diversos autores advierten sobre tres desafíos fundamentales:

\textbf{1. Huella ambiental del hardware:} La implementación masiva de sensores IoT conlleva el riesgo de generar residuos electrónicos (e-waste) debido a la corta vida útil de los componentes en ambientes corrosivos como los de la bioconversión (alta humedad y presencia de amoniaco). Además, la fabricación y operación de estos dispositivos tiene su propia huella de carbono asociada, que debe ser compensada por las eficiencias que el sistema genera \parencite{dharCarbonImpactArtificial2020}.

\textbf{2. Robustez y Sesgo Algorítmico:} Los modelos de aprendizaje automático entrenados en condiciones controladas de laboratorio a menudo fallan al generalizar en entornos rurales reales, donde la variabilidad de los residuos y las condiciones climáticas es alta. Confiar ciegamente en predicciones algorítmicas sin una validación física continua puede llevar a decisiones de manejo erróneas \parencite{klerkxReviewSocialScience2019}.

\textbf{3. Brecha de Implementación:} En contextos del Sur Global, la dependencia de conectividad estable y soporte técnico especializado puede convertirse en una barrera de adopción para pequeños productores. Por tanto, la innovación tecnológica debe equilibrarse con criterios de robustez, bajo costo y mantenibilidad local para ser verdaderamente sostenible.

\section{Estudios Previos sobre la Gestión de Residuos con \textit{BSF}}

En los últimos años, la literatura científica ha validado el potencial de BSF no solo como agente de reducción de residuos, sino como bioreactor eficiente. Estudios fundamentales, como los de \textcite{salamEffectDifferentEnvironmental2022}, han caracterizado la sensibilidad del proceso a variables abióticas, estableciendo que rangos de temperatura entre $25-30\ ^{\circ}\mathrm{C}$ y humedad del sustrato mayores que $60\%$ son críticos para maximizar la tasa metabólica. \textit{Esta dependencia no lineal entre las condiciones ambientales y el rendimiento biológico subraya la necesidad de sistemas de control dinámico, ya que desviaciones menores en estas variables pueden alterar drásticamente la eficiencia de bioconversión.}

La transición hacia modelos predictivos ya cuenta con precedentes importantes que es necesario analizar. En el ámbito del modelado matemático, en \parencite{eriksenDynamicModellingFeed2022} desarrollaron una aproximación mecanística basada en la teoría de Presupuestos Energéticos Dinámicos (DEB), logrando simular el crecimiento larval en función del alimento disponible. Sin embargo, este modelo se centra exclusivamente en la acumulación de biomasa, dejando de lado la predicción dinámica de los flujos gaseosos asociados. Paralelamente, en el dominio del hardware, autores como \textcite{rajRealTimeEstimation2023} y \textcite{duobieneDevelopmentWirelessSensor2022} han implementado prototipos de monitoreo IoT para variables ambientales básicas (temperatura, humedad). Aunque estos trabajos demuestran la viabilidad técnica de la sensorización, se limitan a la vigilancia pasiva de condiciones, careciendo de una capa de inteligencia artificial que integre la variable biológica con la ambiental para predecir anomalías en tiempo real.

Si bien investigaciones recientes han ampliado el horizonte de valorización hacia biocombustibles y bioplásticos \parencite{koyunogluBiofuelProductionUtilizing2024,suryatiLauricAcidBlack2023}, la mayoría de estos trabajos adopta un enfoque de “caja negra”, analizando entradas y salidas sin modelar la dinámica interna del proceso. Persiste, por tanto, una brecha notable: la falta de integración entre los modelos biológicos profundos (tipo EDO/DEB) y la captura de datos en tiempo real (IoT) para la predicción específica de GEI, lo cual limita la capacidad de intervención preventiva en sistemas industriales.

\section{Medición y Dinámica de Emisiones de GEI en la Bioconversión}

La cuantificación de GEI es el pilar para la validación ambiental de la tecnología. Aunque el proceso es predominantemente aeróbico —generando principalmente \ce{CO2} y calor metabólico—, la heterogeneidad del sustrato puede inducir la formación de microambientes anaeróbicos productores de \ce{CH4} y \ce{N2O}, gases con un potencial de calentamiento global significativamente mayor \parencite{bekkerImpactSubstrateMoisture2021a}.

La evidencia cuantitativa actual es dispersa. Por ejemplo, \textcite{jenkinsProcessingPoultryManure2025} reportan reducciones netas de emisiones frente al compostaje debido a ciclos de tratamiento más cortos. En términos de factores de emisión, \textcite{fuhrmannComprehensiveIndustryrelevantBlack2025} reportan $96.5\,\mathrm{g}$ \ce{CO2} por kg de residuo, mientras que \textcite{rossiEstimatingDynamicsGreenhouse2024} presentan valores normalizados por biomasa seca ($7.76$ -- $11.88\,\mathrm{g}$ \ce{CO2}/g larva). \textit{Esta disparidad en las unidades funcionales y la falta de protocolos estandarizados dificultan la comparación directa y el establecimiento de líneas base para mecanismos de compensación de carbono.}

La mayoría de estos estudios se basan en muestreos puntuales (cámaras estáticas) que no capturan los picos transitorios de emisión. Aquí radica una oportunidad para ésta investigación: la implementación de sistemas de monitoreo continuo (IoT) no solo permitiría refinar estos factores de emisión, sino utilizar el \ce{CO2} como indicador directo de la actividad metabólica y el \ce{CH4} como indicador temprano de fallos operativos (anaerobiosis), transformando la medición ambiental en una herramienta de control de procesos \parencite{kamilaris2017review,zhangApplicationBigData2024}.

\section{Antecedentes en Modelamiento Matemático de Procesos de Bioconversión}

El modelamiento matemático constituye la base teórica para transitar de la observación empírica a la predicción cuantitativa en procesos biológicos. En el contexto de la bioconversión con BSF, los modelos mecanísticos permiten desacoplar los fenómenos de crecimiento larval de la dinámica de degradación del sustrato, estableciendo relaciones causales entre la entrada de materia orgánica y la generación de gases de efecto invernadero (GEI), principalmente \ce{CO2} y, en condiciones subóptimas, \ce{CH4}.

La literatura actual se ha centrado predominantemente en modelos deterministas basados en sistemas de Ecuaciones Diferenciales Ordinarias (EDO). Referentes como \textcite{padmanabhaComprehensiveDynamicGrowth2020} adaptaron el marco de Presupuestos Energéticos Dinámicos (DEB), logrando describir con alta precisión los flujos metabólicos a nivel individual (respiración, asimilación y excreción). Posteriormente, \textcite{eriksenDynamicModellingFeed2022} escalaron estos principios a nivel poblacional, incorporando balances de masa para predecir la producción de calor y \ce{CO2}. Sin embargo, estos modelos asumen condiciones de homogeneidad ideal en el sustrato, lo que limita su capacidad para predecir la formación de microambientes anaeróbicos y, por ende, la generación estocástica de metano.

Para abordar la cinética de degradación, otros autores han extrapolado funciones de tipo Monod y Haldane desde la digestión anaerobia \parencite{winAnaerobicDigestionBlack2018, ahlamineMathematicalAnalysisAnaerobic2024}. Si bien estas ecuaciones capturan las no linealidades del crecimiento microbiano y la inhibición por sustrato, su aplicación directa en matrices sólidas porosas (como el lecho de cría de BSF) introduce errores estructurales al no considerar la variabilidad espacial de la temperatura y la humedad \parencite{rossiEstimatingDynamicsGreenhouse2024}.

En la tesis, los modelos mecanísticos actuales (EDO/DEB) son excelentes para describir el comportamiento “promedio” o ideal del sistema (tendencia base), pero carecen de la flexibilidad plástica para adaptarse a perturbaciones no modeladas o variaciones abruptas en la composición del residuo. La literatura evidencia una ausencia de arquitecturas híbridas que utilicen datos de sensores en tiempo real para recalibrar dinámicamente estos parámetros cinéticos, una estrategia necesaria para garantizar la precisión predictiva en escenarios de gestión operativa real.

\section{Avances en IoT y AAA para la gestión sostenible de residuos}

La incorporación de tecnologías basadas en IoT y AAA ha demostrado tener un impacto tangible en la sostenibilidad de los sistemas de gestión de residuos orgánicos. En particular, \textcite{galan-diazCarbonWaterFootprint2024} documentan cómo la implementación de sensores y automatización en granjas de \textit{BSF} permitió una reducción significativa de la huella de carbono y agua, gracias al control preciso de variables operativas como la temperatura, la humedad y el consumo energético. Este enfoque no solo mejora la eficiencia termodinámica del proceso, sino que habilita una trazabilidad digital rigurosa.

De forma complementaria, \textcite{jenkinsProcessingPoultryManure2025} evidencian que la integración de tecnologías inteligentes en el tratamiento de estiércol avícola contribuye a una disminución considerable en las emisiones de \ce{CO2} y \ce{N2O}, reforzando el papel de estas herramientas como catalizadores para la mitigación del cambio climático. En conjunto, estos estudios sugieren que la automatización no es solo una mejora productiva, sino un requisito habilitante para que los sistemas de tratamiento alcancen estándares de neutralidad en carbono y se integren efectivamente en estrategias de economía circular.

\section{Brechas tecnológicas y oportunidades de investigación}


A pesar de estos avances, la implementación operativa de sistemas inteligentes en la bioconversión con \textit{BSF} enfrenta barreras estructurales que limitan su escalabilidad y confiabilidad. La literatura identifica cinco brechas críticas que esta investigación busca abordar:
\begin{itemize}
\item Materialidad y Resiliencia del Hardware: Escasa disponibilidad de sensores de bajo costo capaces de mantener la precisión (calibración estable) en las condiciones agresivas del proceso (HR $>80\%$), lo que suele derivar en datos ruidosos o deriva instrumental \parencite{ahmedAgricultureClimateChange2020}.

\item Interoperabilidad: Falta de estandarización en protocolos de comunicación que impide la integración fluida entre dispositivos IoT heterogéneos y plataformas de gestión ambiental \parencite{vanIntegrationInternetofThingsSustainable2022}.

\item Limitaciones de los Modelos de Caja Negra: Los algoritmos de aprendizaje automático (AAA) puros, aunque potentes, carecen de restricciones físicas o biológicas explícitas. Esto los hace propensos al \textit{overfitting} y reduce su capacidad de generalización ante condiciones no vistas (como cambios de sustrato), a menos que se hibriden con modelos mecanísticos (EDO) \parencite{weichertReviewMachineLearning2019}.

\item Ausencia de Estándares MRV Dinámicos: Inexistencia de metodologías validadas para el Medición, Reporte y Verificación (MRV) en tiempo real, lo que excluye a la BSF de los mercados de carbono regulados \parencite{herrinCollateralDataQuality2021}.

\item Brecha de Implementación Rural: Dificultades para escalar soluciones dependientes de la nube en contextos con baja infraestructura de conectividad, exigiendo arquitecturas de procesamiento en el borde (\textit{Edge Computing}).
\end{itemize}

Estas brechas delinean una oportunidad estratégica para el desarrollo de una arquitectura híbrida compuesta por: (i) una red de sensores robustecida para monitoreo ambiental; (ii) modelos EDO para capturar la dinámica biológica base; (iii) algoritmos de corrección por AAA para adaptarse a la variabilidad real; y (iv) una interfaz de soporte a la decisión. Esta integración apunta a transformar la bioconversión de una caja negra biológica a un proceso transparente, tecnológicamente viable y verificable climáticamente \parencite{kamilaris2017review}.

\section{Brechas estructurales en el modelamiento matemático de la bioconversión}

Aunque el modelamiento matemático ha logrado avances significativos en la descripción bioenergética del crecimiento larval \parencite{padmanabhaComprehensiveDynamicGrowth2020,eriksenDynamicModellingFeed2022}, persisten brechas estructurales que limitan su validez ecológica y operativa. La principal limitación radica en el enfoque determinista y de \textit{estado estacionario} que predomina en la literatura: la mayoría de los modelos asumen condiciones ambientales homogéneas e idealizadas, ignorando los gradientes espaciales de temperatura y oxigenación que ocurren naturalmente en el lecho de sustrato. Esta simplificación impide predecir fenómenos emergentes críticos, como la formación de núcleos anaeróbicos generadores de \ce{CH4} y \ce{N2O}, cuya aparición es estocástica y dependiente de la microestructura del medio \parencite{bekkerImpactSubstrateMoisture2021a,rossiEstimatingDynamicsGreenhouse2024}.

Una segunda brecha crítica es la desconexión entre la teoría (modelo) y la observación (sensor). Los modelos vigentes operan en lazo abierto, incapaces de asimilar datos en tiempo real para corregir sus trayectorias ante perturbaciones no modeladas (ej. cambios súbitos en la composición del residuo). Esta rigidez contrasta con la naturaleza adaptativa de los sistemas biológicos, resultando en errores de predicción que aumentan con el tiempo de simulación. A esto se suma la falta de estandarización en las métricas de validación —con estudios que reportan emisiones en unidades funcionales incomparables—, lo que fragmenta el conocimiento y dificulta la consolidación de factores de emisión robustos para la industria \parencite{fuhrmannComprehensiveIndustryrelevantBlack2025}.

Estas limitaciones delinean una frontera de investigación clara: la necesidad de transitar de modelos puramente mecanísticos a arquitecturas de modelado híbrido. La integración de ecuaciones diferenciales (para la tendencia biológica) con algoritmos de aprendizaje automático alimentados por IoT (para la corrección de errores residuales) surge como la estrategia óptima para dotar al sistema de la capacidad adaptativa necesaria para operar en entornos agroindustriales reales \parencite{kobelskiModelbasedProcessOptimization2024a,liuOptimizingDataPipelines2023}.