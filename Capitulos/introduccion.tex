\chapter{Introducción}

El cambio climático, principalmente resultado de las actividades humanas, se ha convertido en uno de los retos más urgentes de nuestra era \parencite{allanSummaryPolicymakers2023}. Desde la época preindustrial, las concentraciones de gases de efecto invernadero (GEI) como el dióxido de carbono (CO\textsubscript{2}), el metano (CH\textsubscript{4}) y el óxido nitroso (N\textsubscript{2}O) han alcanzado niveles sin precedentes. En 2024, estas concentraciones llegaron a 423.9 partes por millón (ppm) para el CO\textsubscript{2}, 1942 partes por billón (ppb) para el CH\textsubscript{4} y 338.0 ppb para el N\textsubscript{2}O, según las mediciones globales más recientes \parencite{wmoWMOGreenhouseGas2025}. Si bien una concentración atmosférica de CO\textsubscript{2} cercana al 0.042\% puede parecer nominal en términos volumétricos, su relevancia no reside en la abundancia relativa, sino en las propiedades radiativas intrínsecas de la molécula, la cual absorbe y reemite eficazmente la radiación infrarroja térmica. En consecuencia, este incremento acelerado ha alterado el forzamiento radiativo y ha intensificado el calentamiento global, provocando eventos climáticos extremos, alteraciones significativas en los ecosistemas y una rápida pérdida de biodiversidad. 

Dentro de las fuentes antropogénicas de GEI, la gestión inadecuada de residuos orgánicos se destaca como un factor crítico, ya que su descomposición anaeróbica genera grandes cantidades de CH\textsubscript{4}, un gas con un potencial de calentamiento global 28 veces superior al del CO\textsubscript{2} en un horizonte de 100 años. Aunque los océanos y ecosistemas terrestres han actuado históricamente como sumideros naturales, absorbiendo gran parte del CO\textsubscript{2} antropogénico, reportes recientes advierten sobre una reducción en la eficiencia de estos sumideros durante 2024, lo que podría señalar una peligrosa retroalimentación climática. Por tanto, ante los signos de saturación de los mecanismos naturales de absorción y el alto potencial de calentamiento del metano, es urgente desarrollar soluciones innovadoras y sostenibles para mitigar las emisiones actuales, particularmente aquellas vinculadas a la gestión de residuos orgánicos \parencite{ginerFoodLossWaste2025a}.

Las estrategias convencionales para la gestión de residuos orgánicos —como los rellenos sanitarios, la digestión anaerobia y el compostaje— enfrentan limitaciones significativas. Los rellenos sanitarios generan emisiones prolongadas y difícilmente controlables de metano durante años e incluso décadas; la digestión anaerobia, aunque permite capturar biogás, demanda infraestructura compleja y costosa; y el compostaje, a pesar de su mayor accesibilidad, conlleva emisiones de N\textsubscript{2}O y requiere períodos prolongados para lograr la estabilización del material \parencite{kongEvaluatingGreenhouseGas2012,nordahlGreenhouseGasAir2023}. En contextos agroindustriales de países en desarrollo —donde se concentra la mayor parte de los residuos orgánicos sin tratar—, estas alternativas suelen resultar inviables desde el punto de vista técnico o económico. Justamente esta brecha abre una ventana de oportunidad para tecnologías alternativas basadas en procesos biológicos que puedan ser controlados, escalables y adaptados a estas realidades.

Entre las alternativas emergentes, la bioconversión de residuos orgánicos mediante larvas de \textit{Hermetia illucens} (conocida como mosca soldado negra o BSF, por sus siglas en inglés) ha demostrado ser una tecnología prometedora y transformadora. A diferencia de los métodos convencionales, la BSF opera bajo un enfoque de economía circular: transforma residuos orgánicos en biomasa larval rica en proteínas y lípidos y en fertilizante orgánico de alta calidad (\textit{frass}), todo mediante procesos biológicos susceptibles de diseño \parencite{bermudezComprehensiveUtilizationBlack2023,Surendra2016}. Comparada con métodos tradicionales, la bioconversión con BSF presenta ventajas claras: requiere menor infraestructura, opera en tiempos más cortos (3--4 semanas), funciona sin generación de lixiviados contaminantes y genera productos valiosos. Sin embargo, este potencial contrasta con un vacío de información crítica: la mayoría de las implementaciones operan como una ``caja negra'', sin monitoreo ambiental riguroso y sin una cuantificación fiable de los flujos de carbono, lo que impide validar su sostenibilidad real frente al compostaje convencional \parencite{jenkinsProcessingPoultryManure2025}.

Teniendo en cuenta estas limitaciones, se propone la integración de \textbf{tres componentes tecnológicos complementarios}: (1) sensores IoT de bajo costo, (2) modelos matemáticos (EDO) y (3) algoritmos de aprendizaje automático (AAA). Esta convergencia tecnológica no es aditiva, sino \textbf{estructuralmente necesaria} para superar las barreras de cada método por separado: mientras los sensores capturan la realidad física pero carecen de contexto ("ceguera semántica"), y y los modelos matemáticos mecanicistas (caja blanca) aportan coherencia teórica y respeto por las leyes termodinámicas, pero pierden precisión predictiva ante la estocasticidad del comportamiento larval y la heterogeneidad del sustrato sólido, la inteligencia artificial permite cerrar la brecha adaptándose a la estocasticidad del sistema. Es crucial entender que ninguno de estos componentes por sí solo puede validar la sostenibilidad ambiental del proceso; solo su integración híbrida permite transformar datos crudos en evidencia verificable para la toma de decisiones.

El objetivo de esta tesis es desarrollar un sistema híbrido que integre sensores IoT, modelos matemáticos basados en EDO y algoritmos de aprendizaje automático para estimar en tiempo real las emisiones de CO\textsubscript{2} y CH\textsubscript{4} durante la bioconversión de residuos con BSF. El sistema debe: (1) caracterizar dinámicamente las emisiones durante los ciclos productivos; (2) validar predicciones con tolerancia de error aceptable ($<10\%$ MAPE) frente a métodos estándar (cromatografía); y (3) generar herramientas accesibles (dashboard web) que traduzcan datos complejos en información para la toma de decisiones operativas, permitiendo comparar el desempeño ambiental contra una línea base.

El sistema mide: (1) concentraciones de CO\textsubscript{2} y CH\textsubscript{4} mediante sensores NDIR; y (2) variables de proceso (temperatura, humedad, densidad larval estimada). \textbf{Nota sobre el alcance:} No se mide directamente N\textsubscript{2}O en esta etapa por limitaciones de costos en sensores de bajo costo, aunque la arquitectura modular del sistema está diseñada para permitir su integración futura, reconociendo su importancia en el balance global de GEI.

La propuesta, sin embargo, no ignora las complejidades del entorno real. Reconoce la necesidad de gestionar la deriva de los sensores en condiciones de humedad extrema para evitar lecturas erróneas, así como la urgencia de diseñar soluciones robustas ante la falta de datos masivos y la conectividad rural inestable. Además, el uso de inteligencia artificial se somete a un escrutinio ético-ambiental, evaluando si el costo energético del procesamiento y los riesgos de sesgo justifican el beneficio operativo, un balance ineludible en el campo de los Estudios Ambientales.

\section*{Claridad de propósitos: alcance técnico, marco comparativo y responsabilidad contextual}

Esta tesis tiene un alcance técnicamente delimitado pero socialmente situado: desarrollar un sistema híbrido para estimar en tiempo real las emisiones sin pretender sustituir el juicio técnico o normativo. El sistema es, por diseño, una \textit{herramienta descriptiva de soporte a la decisión}, cuyo propósito es proporcionar a productores información cuantitativa para correlacionar decisiones operativas (densidad, sustrato, aireación) con la huella de carbono resultante. El diseño prioriza la accesibilidad y la transparencia, permitiendo generar datos comparables con alternativas como la digestión anaerobia o los rellenos sanitarios.

Reconociendo que la tecnología no es neutral, esta investigación asume una responsabilidad epistemológica explícita. Los algoritmos reflejan los contextos donde se entrenan y rara vez capturan la variabilidad total de los sistemas locales. Por ello, se establecen tres compromisos: (1) \textbf{transparencia algorítmica}, documentando limitaciones y supuestos (IA Explicable); (2) \textbf{soberanía de datos}, permitiendo el acceso y validación por parte de los usuarios locales; y (3) \textbf{compatibilidad con saberes}, asegurando que la tecnología complemente el conocimiento empírico del manejo de residuos. El éxito del sistema no se mide únicamente por métricas de error estadístico, sino por su capacidad para integrarse de manera justa en sistemas agroindustriales reales, fortaleciendo la autonomía técnica del productor.

\subsection*{Aporte del sistema a la gestión ambiental}

Desde la perspectiva de la gestión ambiental, este sistema representa un cambio de paradigma al transitar de tratamientos de residuos basados en la estimación empírica a procesos gobernados por la precisión de los datos. Su principal aporte radica la capacidad de visibilizar el metabolismo del proceso en tiempo real, permitiendo identificar y mitigar picos de emisión de GEI que, en métodos convencionales, pasarían inadvertidos. Al integrar la detección temprana de condiciones anaeróbicas con modelos predictivos, la herramienta no solo optimiza la huella de carbono del tratamiento, sino que otorga trazabilidad y transparencia al ciclo de valorización de residuos, elementos indispensables para validar estrategias de economía circular y cumplimiento normativo en escenarios de crisis climática.