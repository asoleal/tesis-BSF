\chapter{Resultados Esperados}

Conforme al alcance definido y la ventana de ejecución experimental, el proyecto se orienta a la validación de un sistema tecnológico funcional a escala de laboratorio. El propósito central no es solo la medición instrumental, sino la demostración de que una arquitectura híbrida (Sensores + IA + Bioenergética) puede reducir la incertidumbre en el monitoreo de procesos biológicos complejos, estableciendo bases técnicas para la gestión basada en datos y la reducción de la huella de carbono en la agroindustria.A continuación, se detallan los productos técnicos, científicos y ambientales comprometidos:

\section{Prototipo Funcional de Monitoreo IoT}Se entregará un sistema de adquisición de datos operativo y caracterizado, diseñado para mantener la integridad de la señal en el entorno agresivo de la bioconversión (alta humedad relativa y presencia de gases corrosivos).\begin{itemize}\item \textbf{Entregable:} Nodo de sensores integrado (basado en ESP32 u homólogo) capaz de registrar y transmitir telemetría de \ce{CO2}, \ce{CH4}, temperatura y humedad con alta resolución temporal. El hardware incluirá estrategias de preacondicionamiento de señal para mitigar el ruido electrónico inherente al entorno.\item \textbf{Criterio de Éxito:} Continuidad operativa documentada durante ciclos completos de 14 días, con una tasa de pérdida de paquetes de datos $< 5\%$. Se validará la calidad del dato mediante una correlación de Pearson ($R > 0.85$) y un error medio ($MAPE \leq 15\%$) frente a instrumentos de referencia, demostrando la viabilidad del bajo costo pese a la deriva instrumental.\end{itemize}

\section{Modelo Híbrido de Estimación de Estados}Se desarrollará un motor de cómputo que integre la inteligencia artificial con la teoría biológica para superar la imposibilidad de medir la biomasa larval en tiempo real.\begin{itemize}\item \textbf{Entregable:} Un algoritmo validado de Arquitectura en Cascada, compuesto por:\begin{enumerate}\item Un estimador basado en Redes Neuronales que infiera la biomasa larval ($W_w$) a partir de las variables ambientales históricas.\item Un modelo mecanístico (EDO-DEB) que utilice dicha biomasa para predecir la tasa de respiración (\ce{CO2}) esperada.\end{enumerate}\item \textbf{Criterio de Éxito:} El sistema híbrido deberá superar la precisión de una línea base estática (modelo de crecimiento lineal), logrando un coeficiente de determinación ($R^2 \geq 0.70$) en la predicción de la tendencia de \ce{CO2}, demostrando que el sistema captura efectivamente la dinámica de crecimiento de BSF.\end{itemize}

\section{Herramienta de Diagnóstico Ambiental}Se entregará una metodología validada para transformar los datos crudos en indicadores de gestión, útiles para la toma de decisiones operativas y el reporte de sostenibilidad.\begin{itemize}\item \textbf{Entregable (Sistema de Alertas):} Un módulo de detección lógica capaz de identificar eventos de Anaerobiosis mediante la lectura directa de \ce{CH4} y el análisis de divergencia entre el modelo y el sensor.\item \textbf{Entregable (Inventario Dinámico):} Un reporte automatizado que cuantifique la masa total de carbono emitida ($C_{total}$), explicitando la discrepancia ($\Delta$) entre la medición real y la estimación teórica tradicional.\item \textbf{Criterio de Éxito:} Una sensibilidad (\textit{Recall}) $> 80\%$ en la clasificación correcta de eventos de pico de metano, asegurando que el sistema actúe como una barrera efectiva contra emisiones fugitivas no reportadas.\end{itemize}

\section{Activo de Información Científica}Como resultado transversal, se consolidará un Dataset Experimental Estructurado sobre la dinámica de emisiones en la cría de BSF.\begin{itemize}\item \textbf{Entregable:} Base de datos curada y etiquetada que correlaciona variables ambientales, crecimiento de biomasa (ground truth) y emisiones de GEI (\ce{CO2}/\ce{CH4}) en alta frecuencia. Este activo servirá como insumo para futuras investigaciones o para el re-entrenamiento de algoritmos más complejos.\end{itemize}


\section{Interfaz de Visualización y Soporte Operativo}Para garantizar que la información técnica sea accesible para la toma de decisiones, se implementará una capa de visualización de datos de bajo costo computacional.\begin{itemize}\item \textbf{Entregable:} Un Dashboard de Control (basado en plataformas \textit{open-source} como Grafana, Streamlit o ThingsBoard) que despliegue en tiempo real:\begin{enumerate}\item Las curvas de concentración de \ce{CO2} y \ce{CH4}.\item El estado semafórico del sistema (Verde: Eficiente / Rojo: Alerta de Anaerobiosis).\item El valor acumulado de emisiones estimada vs. real.\end{enumerate}\item \textbf{Criterio de Éxito:} La interfaz deberá actualizarse automáticamente con una latencia $< 1$ minuto respecto a la transmisión del sensor, permitiendo la interpretación visual inmediata de las anomalías por parte de un operador no experto.\end{itemize}

\section{Producción Intelectual y Divulgación}Como validación externa de la calidad científica del desarrollo.\begin{itemize}\item \textbf{Entregable:} Un manuscrito científico (tipo \textit{Conference Paper} o artículo de revista indexada) que documente la metodología de calibración de sensores \textit{low-cost} en ambientes de alta humedad o la validación del modelo híbrido para BSF.\item \textbf{Criterio de Éxito:} Sometimiento o aceptación del trabajo en un evento académico o revista del área de Ingeniería Ambiental, Electrónica o Bioeconomía.\end{itemize}