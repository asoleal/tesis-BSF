\chapter{Objetivos de la Investigación}

\section{Objetivo General}

Desarrollar un sistema híbrido a escala de laboratorio —integrado por modelado mecanístico (EDO), sensores IoT de bajo costo y algoritmos de aprendizaje automático— para la estimación de la tasa de producción de \ce{CO2} y la detección de eventos de emisión de \ce{CH4} en la bioconversión de residuos con \textit{Hermetia illucens}, con el fin de generar métricas de desempeño ambiental aplicables a la gestión de residuos agroindustriales.

\section{Objetivos Específicos}

\begin{enumerate} \item \textbf{Instrumentación:} Configurar y calibrar un módulo de sensores NDIR de bajo costo para la medición de \ce{CO2} y \ce{CH4}, determinando su precisión y desviación (MAPE) frente a un instrumento de referencia bajo las condiciones de temperatura y humedad propias del cultivo de larvas.

\item \textbf{Modelado Matemático:} Adaptar y parametrizar un modelo de EDO basado en la bioenergética de \textit{Hermetia illucens} para estimar la tasa de respiración (\ce{CO2}), evaluando su capacidad predictiva en ventanas temporales durante un ciclo de bioconversión de 15 días.

\item \textbf{Integración Híbrida:} Desarrollar un algoritmo de estimación de estados mediante redes neuronales que infiera la biomasa larval a partir de datos operativos, integrando esta variable como entrada dinámica al modelo mecanístico para mejorar la precisión de la estimación de \ce{CO2}.

\item \textbf{Métricas Ambientales:} Validar la utilidad del sistema para la gestión ambiental mediante la detección de eventos de anaerobiosis (\ce{CH4}) y la cuantificación del carbono emitido, contrastando estos indicadores dinámicos frente a estimaciones estáticas tradicionales.
\end{enumerate}