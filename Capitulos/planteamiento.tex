\chapter{Planteamiento del Problema}

En el contexto Tecnológico y Biológico, la gestión de la fracción orgánica de los residuos sólidos municipales (FORSU) enfrenta limitaciones estructurales en sus rutas convencionales. Los rellenos sanitarios operan como reactores anaerobios no controlados de larga duración; el compostaje, aunque accesible, requiere extensos tiempos de estabilización y conlleva riesgos de emisiones de óxido nitroso (N\textsubscript{2}O); y la digestión anaerobia, pese a su ventaja energética, demanda inversiones de capital, a menudo inviables para la escala agroindustrial media.

En este escenario, la bioconversión mediante larvas de \textit{Hermetia illucens} (BSF) se ha consolidado como una alternativa de alta eficiencia. A diferencia de los métodos pasivos, este es un sistema biológico acelerado: durante su fase larval (aprox. 14 días), el insecto incrementa su masa corporal hasta 4000 veces mediante un metabolismo voraz y exotérmico que ocurre en simbiosis con una microbiota activa. Este proceso transforma la materia orgánica en biomasa (proteína) y fertilizante (\textit{frass}) con baja huella hídrica, posicionándose como una herramienta clave para la economía circular \parencite{Surendra2016,bermudezComprehensiveUtilizationBlack2023}.

Sin embargo, a pesar de estas ventajas biológicas, la ausencia de cuantificación dinámica y rigurosa de las emisiones de gases de efecto invernadero (GEI), particularmente CH\textsubscript{4} y CO\textsubscript{2}, genera una incertidumbre crítica. Actualmente, la tecnología opera mayoritariamente como una ``caja negra'': se conocen los insumos y productos, pero se ignora la dinámica de los flujos gaseosos intermedios, basándose en factores de emisión estáticos o extrapolaciones de otros procesos (compostaje) que no reflejan la realidad metabólica de la BSF \parencite{fuhrmannComprehensiveIndustryrelevantBlack2025,rossiEstimatingDynamicsGreenhouse2024}. 

Esta fragmentación del conocimiento tiene consecuencias técnicas, operativas y de gobernanza climática:

\textbf{Primero}, al no existir factores de emisión específicos para BSF en las metodologías del IPCC ni en el GHG Protocol, cualquier balance de carbono se basa en supuestos no verificados. Esto introduce sesgos sistemáticos que pueden subestimar significativamente las emisiones de CH\textsubscript{4}, especialmente bajo condiciones de manejo subóptimas. La consecuencia no es solo técnica, sino ética: se corre el riesgo de asignar beneficios climáticos no reales a una tecnología que, sin monitoreo, podría superar las emisiones del compostaje en escenarios de anaerobiosis no controlada (Greenwashing involuntario).

\textbf{Segundo}, la imposibilidad de detectar en tiempo real picos de CH\textsubscript{4} impide identificar fallas operativas críticas —como sobrepoblación larval, exceso de humedad o deficiente aireación— que generan microambientes anaeróbicos. Estas condiciones no solo aumentan las emisiones de GEI, sino que reducen la eficiencia de conversión, incrementan la mortalidad larval y degradan la calidad del \textit{frass}, impactando directamente la viabilidad económica del sistema. Medir las emisiones es, por tanto, una herramienta de control de proceso.

\textbf{Tercero}, la carencia de sistemas de Medición, Reporte y Verificación (MRV) alineados con estándares internacionales constituye una barrera estructural para el acceso a mecanismos de financiación climática. Según el Banco Mundial, los mercados de carbono modernos exigen trazabilidad cuantitativa y verificable \parencite{worldbankStateTrendsCarbon2023}. Normativas como el \textit{Verified Carbon Standard} (VCS) estipulan que la cuantificación debe ser "real, medible y permanente" \parencite{verraVCSStandardV472024}. Sin datos empíricos robustos, la industria de insectos no puede certificar sus reducciones a escala industrial \parencite{smetanaSustainableUseHermetia2019}.

Esta brecha se ve agravada por la escasa integración de tecnologías de monitoreo con modelos predictivos. La mayoría de las implementaciones actuales carecen de retroalimentación en tiempo real, operando con parámetros fijos que ignoran la dinámica no lineal del proceso biológico \parencite{kamilaris2017review,liuOptimizingDataPipelines2023}.

Frente a este escenario, esta tesis propone un sistema híbrido que transforme la cuantificación de emisiones en una herramienta de gestión ambiental y operativa. Al estimar de forma continua y precisa las emisiones de CO\textsubscript{2} y CH\textsubscript{4}, el sistema permitirá: (1) generar la evidencia empírica necesaria para establecer factores de emisión específicos y alinear la BSF con marcos climáticos globales; (2) optimizar el proceso mediante la detección temprana de ineficiencias a partir de los gases medidos; y (3) construir una trazabilidad verificable que permita hacia el futuro la participación de los productores agroindustriales en economías circulares con credenciales climáticas robustas.