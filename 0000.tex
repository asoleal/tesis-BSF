% Esta es la Plantilla UNAL en LaTeX
\documentclass[10pt,spanish,fleqn,openany,twoside,letterpaper]{book}

% --- Codificación (solo para pdflatex) ---
\usepackage[T1]{fontenc}
\usepackage[utf8]{inputenc}
\usepackage{lmodern}

% --- Carga modular (¡solo una vez!) ---
\usepackage{preambulo/paquetes_comunes}

% --- Metadatos ---
% Definición de comandos (inicialización)
\newcommand{\studentname}{}
\newcommand{\submissiondate}{}
\newcommand{\academictitle}{}
\newcommand{\resgroupone}{}
\newcommand{\resgrouptwo}{}
\newcommand{\researchtopic}{}
\newcommand{\thesisname}{}
\newcommand{\thesisnameeng}{}
\newcommand{\thesisnamelang}{}
\newcommand{\director}{}
\newcommand{\directortitle}{}
\newcommand{\codirector}{}
\newcommand{\codirectortitle}{}
\newcommand{\issuedate}{}
\newcommand{\palabrasclave}{}
\providecommand{\keywords}{} % <--- CAMBIO AQUÍ: providecommand evita el conflicto
\newcommand{\schlusselworter}{}
\newcommand{\palavraschave}{}
\newcommand{\sede}{}
\newcommand{\department}{}
\newcommand{\departmenttwo}{}
\newcommand{\faculty}{}
\newcommand{\university}{Universidad Nacional de Colombia}

% Asignación de valores
\renewcommand{\studentname}{John Jairo Leal Gómez}
\renewcommand{\thesisname}{Nombre del trabajo o tesis}
\renewcommand{\thesisnameeng}{Nombre del trabajo o tesis en inglés}
\renewcommand{\issuedate}{Año entrega}
\renewcommand{\submissiondate}{Fecha entrega}
\renewcommand{\director}{Dr. Luis Octavio González Salcedo}
\renewcommand{\directortitle}{Profesor Titular}
\renewcommand{\codirector}{Dr. Luis Fernando Mejia }
\renewcommand{\codirectortitle}{Indicar si es Profesor Titular/Asociado}
\renewcommand{\academictitle}{Magíster (M, MSc) o Doctor (PhD) - Modalidad }
\renewcommand{\resgroupone}{Grupo A (Sigla Grupo Investigación 01) }
\renewcommand{\resgrouptwo}{Grupo B (Sigla Grupo Investigación 02) }
\renewcommand{\researchtopic}{Línea}
\renewcommand{\sede}{Sede de la Universidad}
\renewcommand{\department}{Departamento}
\renewcommand{\departmenttwo}{}
\renewcommand{\faculty}{Facultad}
\renewcommand{\palabrasclave}{Palabras clave}
\renewcommand{\keywords}{Keywords}


% --- Paquetes específicos de la tesis (no en paquetes_comunes) ---
\usepackage{layout}
\usepackage[hidelinks]{hyperref}
\usepackage{fancyhdr}
% ... resto de configuración (titlesec, appendix, etc.)

\begin{document}

%Nombres y formatos de títulos, tablas y figuras
%Use \sffamily para dejar con letra Sans Serif, sin etiqueta queda LaTeX clásico
\renewcommand{\listfigurename}{\sffamily Lista de figuras}
\renewcommand{\listtablename}{\sffamily Lista de tablas}
\renewcommand{\contentsname}{\sffamily Contenido}
\renewcommand{\chaptername}{\sffamily Capítulo}
\renewcommand{\tablename}{\scriptsize \centering \textbf{Tabla}}
\renewcommand{\figurename}{\scriptsize \centering \textbf{Figura}}
\renewcommand{\appendixname}{\sffamily Anexo}

%Cambia el nombre de la sección de referencias
\renewcommand{\bibname}{\sffamily Referencias Bibliográficas}

%Páginas de Presentación del documento - No modificar esto se hace automáticamente
{\newpage
\thispagestyle{empty}
\begin{center}
\begin{figure}
\centering
\epsfig{file=00Figuras/00f00EscudoUN2016,scale=1}%
\end{figure}
\vspace{2.5cm}
\textbf{\Huge \thesisname} \\ 
\vspace{2.5cm}
\textbf{\Large \studentname} \\
\vspace{5.0cm}
\faculty \\ \department \\
\sede, Colombia (Nota: ciudad sede donde se gradúa [borrar esta nota])\\
\issuedate
\newpage 
\thispagestyle{empty}
\vspace{2.0cm}
\textbf{\Huge \thesisname} \\
\vspace{2.0cm}
\textbf{\Large \studentname} \\
\vspace{2.0cm}
\small Tesis presentada como requisito parcial para optar por el título de: \\
{\bfseries \academictitle}\\
\vspace{2.0cm}
\textbf{Director(a):} \\
\director \\
\directortitle \; - \departmenttwo \\
\faculty \\
\university \\ 
\vspace{0.5cm}
\textbf{Codirector(a):} \\
\codirector \\
\codirectortitle \; - \department \\
\faculty \\
\university 
\vspace{1.5cm} \\
\textbf{Línea de investigación:} \\ 
\researchtopic\\
\textbf{Grupo de investigación:} \\
\resgroupone \\
\resgrouptwo \\
\vspace{1.5cm} 
\university \\
\faculty \\
\department \\
\issuedate
\end{center}

% Dedicatorias
\newpage
\thispagestyle{empty}
\begin{flushright}
\begin{minipage}{12.5cm}
\noindent
\\[10em]
%Modificar la cita que se quiere agregar
{\Large Cita 01.}
\\[3em]
Autor
\\ \textit{Fuente}
\\[10em]
%Para anular la adición de una segunda cita anule las siguientes lineas desde acá mediante comentario (%)
{\Large \textit{Wenn du es nicht einfach erkl\"{a}ren kannst, hast du es nicht genug verstanden} - Si no eres capaz de explicar algo claramente, es que aún no lo has entendido lo suficiente.}
\\[3em]
Albert Einstein
%Hasta acá!
\end{minipage}
\end{flushright} 

% Declaracíon de originalidad del texto y del contenido
% No modificar, se hace automáticamente con los comandos ya definidos
\newpage
\chapter*{\sffamily Declaración}
\par Me permito afirmar que he realizado ésta tesis de manera autónoma y con la única ayuda de los medios permitidos y no diferentes a los mencionados el presente texto. Todos los pasajes que se han tomado de manera textual o figurativa de textos publicados y no publicados, los he reconocido en el presente trabajo. Ninguna parte del presente trabajo se ha empleado en ningún otro tipo de tesis. 
\\[1em]
\sede., \submissiondate
\\[6em]
\rule{6cm}{0.5pt}\\
\studentname
}

%Páginas preámbulo, listado de figuras, tablas y tabla de contenido
{\pagestyle{plain} \pagenumbering{roman}
\setlength{\parskip}{1mm}
\include{contenido/00Agradecimientos}
% Comentar las dos lineas de abajo con % en caso que no se requieran abreviaturas y resumen en el trabajo
\include{contenido/00Abreviaturas}
\newpage
\chapter*{\sffamily Resumen}
\addcontentsline{toc}{chapter}{Resumen}%
\textbf{\Huge \ Título en español } \\
\par Texto del resumen.
\\[2cm]
\textbf{Palabras clave:} \palabrasclave

\newpage 
\chapter*{\sffamily Abstract}
\addcontentsline{toc}{chapter}{Abstract}%
\textbf{\Huge \ Título en inglés} \\
\par Abstract text.
\\[2cm]
\textbf{Keywords:} \keywords

%\newpage 
%\chapter*{\sffamily Zusammenfassung}
%\addcontentsline{toc}{chapter}{Zusammenfassung}%
%\par Zusammenfassung texte.
%\par 
%\\[2cm]
%\textbf{Schlüsselwörter:} \schlusselworter
% Dejar esta parte así para que genere correctamente la página de la tabla de contenido
\addcontentsline{toc}{chapter}{Lista de figuras}
\listoffigures
\clearpage
\addcontentsline{toc}{chapter}{Lista de tablas}
\listoftables
\clearpage
\addcontentsline{toc}{chapter}{Contenido}
\tableofcontents
\clearpage
}

{\pagenumbering{arabic}
\setlength{\parskip}{\baselineskip}
%Incluir secciones del documento de aqui en adelante
%Use \include para incluir desde una página nueva e \input para incluir sin salto de página
\include{contenido/00Intrucciones} % Anular esta linea con comentario de ser necesario
\include{contenido/00HipotesisPlanteamiento}
\include{contenido/00Objetivos}
\include{contenido/01Seccion01}
\include{contenido/02Seccion02}
\include{contenido/03Seccion03}
\include{contenido/04Seccion04}
\include{contenido/05Seccion05}
\include{contenido/06Seccion06}
\include{contenido/07Seccion07}

%Inicio del apéndice o anexos
\begin{appendix}
\include{contenido/08Apendice01}%
\end{appendix}

%Permite visualizar la bibliografía en la tabla de contenido
%Cambie el nombre a Bibliografía o Literatura Citada en la siguiente línea de ser preciso
\addcontentsline{toc}{chapter}{Referencias Bibliográficas} 

\let\OLDthebibliography=\thebibliography
\def\thebibliography#1{\OLDthebibliography{#1}}
{\scriptsize
\pagestyle{plain}
% Nombre del documento donde se almacenan las referencias
\bibliographystyle{dtvstyle}
\bibliography{Referencias}
\nocite{*}
% Inserta un página adicional al final en blanco 
%\cleardoublepage
% Para NO insertar una página adicional al final usar \clearpage
\clearpage
}}

\end{document}
